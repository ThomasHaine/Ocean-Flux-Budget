\documentclass[10pt]{amsart}
\usepackage{geometry}                % See geometry.pdf to learn the layout options. There are lots.
\geometry{letterpaper}                   % ... or a4paper or a5paper or ... 
%\geometry{landscape}                % Activate for for rotated page geometry
\usepackage[parfill]{parskip}    % Activate to begin paragraphs with an empty line rather than an indent
\usepackage{graphicx}
\usepackage{amssymb}
\usepackage{amsaddr}
\usepackage{epstopdf}
\usepackage{natbib}

\DeclareGraphicsRule{.tif}{png}{.png}{`convert #1 `dirname #1`/`basename #1 .tif`.png}

%\usepackage{setspace}
%\usepackage{etoolbox}
%\AtBeginEnvironment{quote}{\par\singlespacing\small}

\newcommand{\Cp}{C_{\mbox{\tiny p}}}
\newcommand{\degreeCelsius}{$^{\rm o}$C}
\newcommand{\Qm}{Q^{\mbox{\tiny m}}}
\newcommand{\Qadv}{Q^{\mbox{\tiny adv}}}
\newcommand{\surf}{\Omega}
\newcommand{\Thref}{\Theta_{\mbox{\tiny ref}}}
\newcommand{\Sref}{S_{\mbox{\tiny ref}}}
\newcommand{\MS}{M_{\mbox{\tiny S}}}
\newcommand{\V}{{\mathcal V}}
\newcommand{\Heat}{{\mathcal H}^{\Thref}}
\newcommand{\LFC}{\V^{\Sref}}


\title{Ocean Fluxes and Budgets}
\author{Thomas W. N. Haine}
\address{Earth \& Planetary Sciences, Johns Hopkins University, Baltimore, MD USA}
\email{Thomas.Haine@jhu.edu}
\date{\today}                                           % Activate to display a given date or no date

\begin{document}
\maketitle


\section*{Abstract}

An essential task in physical oceanography is to construct budgets of conserved quantities, like heat, salt and seawater mass, and the masses of trace chemicals. 
In the Arctic and subArctic oceans, for example, such budgets are routinely used to diagnose and understand the effects of natural variations and anthropogenic climate change on temperature and salinity.
Traditionally, temperature variability is analyzed using a budget of heat fluxes relative to a reference temperature. 
Similarly, salinity variability is analyzed using so-called freshwater fluxes relative to a reference salinity.
Well-documented pitfalls exist in the interpretation of these heat and freshwater fluxes, however.
Yet, despite being well-documented, these pitfalls are not universally understood or accepted.

This contribution aims to improve understanding of, and to promote best-practices in, the interpretation of heat and freshwater fluxes, and the construction of their budgets.
The contribution consists of:
(i) A free, open-source, interactive, pedagogical software application called the Ocean-Flux-Budget tool.
(ii) A tutorial YouTube video demonstrating the Ocean-Flux-Budget tool, the pitfalls mentioned above, and suggested workarounds.
(iii) A document explaining the issues, with references to the original literature, and proposed best practices.
To access these resources visit \texttt{https://github.com/ThomasHaine/Ocean-Flux-Budget}.
At this website you can also seek advice, ask questions, and help refine understanding of ocean fluxes and budgets.




\section{Introduction}
This document discusses how to form budgets for conserved quantities in the ocean, like heat and salt. 
The use of reference temperatures and salinities, and the associated ambiguous fluxes of heat and freshwater, are explored.
Equations are developed that highlight the ambiguities and pitfalls in interpretation.
Guidance on how to avoid the pitfalls is offered.
The discussion is based on primary literature by \cite{schauer&besczcynska-moller09,schauer&losch19,bacon_etal15} and \cite{tsubouchi_etal12}.
In a model context, \cite{piecuch_etal17} is also useful.
One prominent application of these concepts is to the Arctic and subArctic Oceans, although the discussion is generic.

Consider the budgets of mass, volume, salt mass, heat (potential enthalpy\footnote{Following \cite{mcdougall03} and \cite{ioc10}.}), and liquid freshwater content for a fixed control volume $V$.
The control volume has bounding surface $\surf$, which includes the sea surface and the sea bed and vertical side walls. 
The arguments below are easily modified to relax these assumptions on the shape of $V$ and $\surf$.
We follow and extend \cite{mcdonagh_etal} (especially their Appendix A).

\section{Seawater mass budget}
\label{sect:mass}
The seawater mass budget is
\begin{align}
\frac{d M}{d t} & = - \int_\surf \rho \, {\bf u} \cdot {\hat{\bf n}} \; d \surf ,
\label{eqn:mass_budget}
\end{align}
% E.g., this is the volume integral of (35) in Nurser & Griffies (2019).
where ${\bf u}$ is the (three-dimensional) vector seawater velocity and ${\hat{\bf n}}$ is the unit outward normal vector to $\surf$.
The seawater mass is 
\begin{align}
M & = \int_V \rho \; d V ,
\end{align}
with (variable) {\it in-situ} seawater density $\rho$.
At the ocean free surface, $z = \eta(x,y,t)$, vertical transport arises from the boundary mass fluxes of precipitation, $P$, evaporation, $E$, river runoff, $R$, sea-ice melt  $M_\mathrm{ice}$, and sea-ice formation $F_\mathrm{ice}$
\begin{equation}
 \rho \, w(x, y, z = \eta, t) \equiv -\Qm  = -(P-E+R+M_\mathrm{ice} - F_\mathrm{ice}),
\label{eqn:surface-mass-fluxes}
\end{equation}
with the boundary mass flux, $\Qm$, having dimensions of mass per horizontal area per time, and with the minus sign indicating that $\Qm >0$ for mass entering the ocean. 

Mass fluxes across the sea floor (e.g., at hydrothermal vents) are similarly included, although here they are neglected for simplicity.

Now consider that the vertical side walls consist of a set of $N$ rectilinear faces, $\surf_i$, enumerated with index $i \in 1 \cdots N$.
These faces are the abstraction of vertical sections across gateway straits, like Davis Strait and Bering Strait.
Thus, (\ref{eqn:mass_budget}) gives
\begin{align}
\frac{d M}{d t} & = \int_{\surf_s} \Qm \; d\surf_s  - \sum_{i=1}^N \int_{\surf_i} \rho \, {\bf u} \cdot {\hat{\bf n}}  \; d \surf_i ,
\label{eqn:mass_budget_with_faces}
\end{align}
where $\surf_s$ denotes the sea surface.
Physically, (\ref{eqn:mass_budget_with_faces}) states that the mass in the control volume changes according to the sum of the mass fluxes across the sea surface and all of the vertical side walls.

\section{Seawater Volume budget}
\label{sect:volume}

The volume of seawater is 
\begin{align}
\V  = \int_V \; dV 
\end{align}
(where $\V $ is the volume in meters cubed, and $V$ is the control volume occupied by seawater).
In the Boussinesq approximation (which is accurate for Earth's ocean; see \citealt{klinger&haine19}, Chapter 1), seawater density is constant for the purposes of budgeting mass.
Thus, 
\begin{align}
\V  = \frac{M}{\rho_0} ,
\end{align}
where $\rho_0$ is the constant characteristic density.
Therefore, volume is a conserved, extensive property\footnote{\label{fn:extensive} An extensive property, like mass, is additive when subsystems are combined. In contrast, an intensive property, like temperature, is not additive in this way.} and the volume budget is simply related to the mass budget in equations (\ref{eqn:mass_budget})--(\ref{eqn:mass_budget_with_faces}).
Without making the Boussinesq approximation, seawater density varies.
Then, mass is still a conserved, extensive variable (section \ref{sect:mass}), but seawater volume is not.
Therefore, volume does not possess an unambiguous physical budget.
In this way, the mass budget is more fundamental.

\section{Seawater Heat budget}
\label{sect:heat}

The heat budget is
\begin{align}
\frac{d {\mathcal H}}{d t} & = - \int_\surf \left[ \Cp \rho \left( {\bf u} \Theta + {\bf F} \right) + {\bf F}_{r} \right] \cdot {\hat{\bf n}} \; d \surf ,
\label{eqn:heat_budget}
\end{align}
where  $\Cp=3991.87$~Jkg$^{-1}$\degreeCelsius$^{-1}$ is the constant heat capacity, $\Theta$ is the Conservative Temperature, and 
\begin{align}
{\mathcal H} & =  \Cp \int_V\rho \Theta \; d V 
\end{align}
is the heat content for the volume $V$.
In (\ref{eqn:heat_budget}), ${\bf F}$ is the flux of heat due to subgridscale processes, typically parametrized as
\begin{align}
{\bf F} & = - \mathbb{K} \cdot \nabla \Theta, 
\label{eqn:flux-gradient-relation}
\end{align}
with $\mathbb{K}$ a second order transport tensor (which could be the molecular diffusivity, or an eddy diffusivity with symmetric and anti-symmetric parts).
The sensible (conductive) heat flux\footnote{Some authors use ``flux'' to mean air/sea exchange, and ``transport'' to mean horizontal advective exchange across a vertical section. Here, ``flux'' covers both possibilities: where necessary, we use qualifiers, like ``advective,'' ``diffusive,'' and ``radiative.''} across the sea surface is represented by this ${\bf F}$ term, for example.
Also in (\ref{eqn:heat_budget}), ${\bf F}_{r}$ is the flux of heat due to radiation (at all wavelengths) and ${\bf F}_{r} \cdot {\hat {\bf n}} < 0$ means ocean warming.

The sea-surface mass flux affects the heat budget \eqref{eqn:heat_budget} as the flux (across surface area $\Delta \Omega _s$) carries heat
\begin{equation}
 \left[ \frac{ d\mathcal{H} }{dt} \right]_{ P-E+R+M_{\mathrm{ice}} - F_{\mathrm{ice}}}  
= \; \Qadv \, \Delta \Omega _s .
\label{eqn:surface-flux-p-e+r}
\end{equation}
Here,
\begin{align}
\Qadv = \Cp  (P\, \Theta^{\,\mathrm{precip}}  - E\, \Theta_s +  R\, \Theta^\mathrm{runoff}  +
M_{\mathrm{ice}} \, \Theta^\mathrm{melt} - F_{\mathrm{ice}} \, \Theta^\mathrm{freeze})
\label{eqn:Qadv}
\end{align} 
defines the advective surface heat flux, and $\Theta^{\,\mathrm{precip}}$ is the Conservative Temperature of the precipitation, $\Theta_s$ is the surface ocean temperature,  $\Theta^{\,\mathrm{runoff}}$ is the temperature of the runoff, $\Theta^\mathrm{melt}$ is the temperature of the melt, and $\Theta^\mathrm{freeze}$ is the local freezing temperature. 

Heat fluxes across the sea floor (e.g., at hydrothermal vents and conductive geothermal processes) are similarly included, although here they are neglected for simplicity.

Consider again the case of $N$ vertical rectilinear faces (gateway straits): (\ref{eqn:heat_budget}) gives
\begin{align}
\frac{d{\mathcal H}}{dt} & = \int_{\surf_s} \Qadv - {\bf F}_{r} \cdot {\hat{\bf n}} \; d\surf_s  - \Cp \sum_{i=1}^N \int_{\surf_i}  \rho \left( {\bf u} \Theta + {\bf F} \right)   \cdot {\hat{\bf n}}  \; d \surf_i ,
\label{eqn:heat_budget_with_faces}
\end{align}
which states physically that the total heat content in $V$ changes according to the sum of all the boundary sources of heat.

Now consider the effect on the heat budget if all temperatures are measured relative to a (constant) reference temperature $\Thref$.
The heat content relative to $\Thref$ is
\begin{align}
\Heat & \equiv  \Cp \int_V\rho \left( \Theta - \Thref \right)  \; d V  ,
\end{align}
so
\begin{align}
\frac{1}{\Cp}\frac{d\Heat}{dt} &= \frac{1}{\Cp} \frac{d {\mathcal H}}{dt} - \Thref \frac{d M}{d t} , \nonumber \\
&= \int_{\surf_s} P (\Theta^{\mathrm{precip}} - \Thref)  - E (\Theta_s  - \Thref) +  R (\Theta^\mathrm{runoff}  - \Thref)  +
M_{\mathrm{ice}}  (\Theta^\mathrm{melt} - \Thref) - F_{\mathrm{ice}}  (\Theta^\mathrm{freeze}  - \Thref)  \; d\surf_s  \nonumber \\ 
&- \frac{1}{\Cp} \int_{\surf_s} {\bf F}_{r} \cdot {\hat{\bf n}} \; d\surf_s  
- \sum_{i=1}^N \int_{\surf_i} \rho \left[ {\bf u} (\Theta  - \Thref)+ {\bf F} \right]   \cdot {\hat{\bf n}}  \; d \surf_i . 
\label{eqn:heat_budget_with_faces_Thref0}
\end{align}
Hence, rearranging and using (\ref{eqn:mass_budget_with_faces}),
\begin{align}
\frac{d\Heat}{dt} & = \int_{\surf_s} \Qadv - {\bf F}_{r} \cdot {\hat{\bf n}} \; d\surf_s  - \sum_{i=1}^N \int_{\surf_i} \Cp \rho \left( {\bf u} \Theta + {\bf F} \right)   \cdot {\hat{\bf n}}  \; d \surf_i \nonumber \\
&- \Cp \Thref \int_{\surf_s} P  - E +  R  + M_{\mathrm{ice}} - F_{\mathrm{ice}}  \; d\surf_s + \Cp \Thref \sum_{i=1}^N \int_{\surf_i} \rho {\bf u}    \cdot {\hat{\bf n}}  \; d \surf_i , \nonumber \\
\implies
\frac{d\Heat}{dt} & = \int_{\surf_s} \Qadv - {\bf F}_{r} \cdot {\hat{\bf n}} \; d\surf_s  - \Cp \sum_{i=1}^N \int_{\surf_i} \rho \left( {\bf u} \Theta + {\bf F} \right)   \cdot {\hat{\bf n}}  \; d \surf_i - \Cp \Thref \frac{dM}{dt} .
\label{eqn:heat_budget_with_faces_Thref}
\end{align}
This equation coincides with (\ref{eqn:heat_budget_with_faces}) for $\Theta_{\rm ref} = 0$, and otherwise generalizes it.

From equation (\ref{eqn:heat_budget_with_faces_Thref}), the following conclusions can be drawn:
\begin{enumerate}
\item The heat budget, meaning $d\Heat/ dt$, is unambiguous (has physical significance because it does not depend on the reference temperature) provided the seawater mass is constant, $dM / dt = 0$.
Moreover, the importance of the individual terms in (\ref{eqn:heat_budget_with_faces_Thref}) setting $d\Heat/ dt$ is unambiguous provided the seawater mass is constant.
\item Computation of advective heat fluxes, $\Heat_i$, across individual faces, 
\begin{align}
\Heat_i & = \Cp \int_{\surf_i} \rho {\bf u} (\Theta  - \Thref)  \cdot {\hat{\bf n}}  \; d \surf_i  , \nonumber
\end{align}
depends on the arbitrary reference temperature $\Thref$ if a non-zero mass flux crosses the face, and therefore has no physical significance.
The advective heat flux $\Heat_i$ is a linear combination of the advective mass flux and the covariance between the velocity and temperature.
Changing the reference temperature changes the importance of these two terms in the linear combination. 

In particular, when there is non-zero mass flux across the face (or faces):
\begin{enumerate}
\item Changing $\Thref$ changes $\Heat_i$. 
Thus, $\Heat_i$ has no physical significance because it depends on the arbitrary value of $\Thref$.
\item For time-varying ${\bf u}, \Theta$, changing $\Thref$ changes $\Heat_i$ by an arbitrary multiple of the mass flux time series. Therefore, there is no physical significance to the correlation between $\Heat_i$ and the mass flux time series across the face.
\item Comparison of the relative importance of the heat flux across the same face for different times depends on the reference temperature. Therefore, it has no physical significance. 
\item The trend in an advective heat flux time series across a face has no physical significance. 
\item The extrema (maximum, minimum) in an advective heat flux time series across a face have no physical significance
\item Comparison of the relative importance of heat fluxes across different faces (e.g., Fram Strait versus Davis Strait for the Arctic heat budget) depends on the reference temperature. Therefore, it has no physical significance.
\end{enumerate}
\item Comparison of advective heat fluxes $\Heat_i$ across faces from models with each other, and from models with observations, is a legitimate test of model realism. The test is legitimate in the sense that it's necessary (but not sufficient) for a model to be realistic to produce the same heat flux across a particular strait as seen in observations. 
\item The average temperature over the control volume $V$ is
\begin{align}
\overline{\Theta} \equiv \frac{{\mathcal H}}{\Cp M} . \nonumber
\end{align}
Therefore, provided the seawater mass $M$ is constant, the change in average temperature can be unambiguously determined from the individual terms in (\ref{eqn:heat_budget_with_faces_Thref}).CHECK!@!

\end{enumerate}

\section{Salt mass budget}
\label{sect:salt}

The salt budget\footnote{The arguments in this section apply to the masses of other dissolved constituents, like trace chemicals.} is
\begin{align}
\frac{d \MS}{d t} & = - \int_\surf \rho \, \left( {\bf u} S_A + {\bf F}_S \right) \cdot {\hat{\bf n}} \; d \surf ,
\label{eqn:salt_budget}
\end{align}
% This is the volume integral of (40) in Nurser & Griffies (2019).
where $S_A$ is the absolute salinity.
The salt mass is 
\begin{align}
\MS & = \int_V \rho  S_A \; d V .
\label{eqn:salt_mass}
\end{align}
In (\ref{eqn:salt_budget}), ${\bf F}_S$ is the flux of salt due to subgridscale processes:
\begin{align}
{\bf F}_S & = - \mathbb{K} \cdot \nabla S_A, 
\label{eqn:flux-gradient-relation_salt}
\end{align}
as for heat in (\ref{eqn:flux-gradient-relation}), although in principle the diffusion tensors for heat and salt can differ.
Notice that no equivalent surface radiative flux term to ${\bf F}_{r}$ appears in (\ref{eqn:salt_budget}).
This is a tacit assumption that salt mass cannot enter or leave the ocean, although exchange with sea ice is included.
Exchange with sea ice is included in (\ref{eqn:salt_budget}) via the ${\bf u} S_A \cdot {\hat{\bf n}}$ term.
Also note that the salt mass (and hence the salinity) is a positive definite quantity. 
Heat is different in this respect, because it has no well-defined zero value.

For these reasons, interpretation of the salt budget ($d \MS / d t$ in (\ref{eqn:salt_budget})) is unambiguous: it doesn't depend on a reference salinity and it doesn't require a zero total mass tendency.
The role of different processes can be obscure, however.
For instance, an ocean region can freshen (decreased average salinity) because of increased precipitation with no change in salt mass. 
Similarly, salt mass can change because of changed advective or diffusive salt fluxes across the boundaries, but not because of changed precipitation.
These obscurities are clarified by recalling how the (absolute) average salinity equals the ratio of salt mass to seawater mass:
\begin{align}
\overline{S_A} = \frac{\MS}{M} .
\label{eqn:average_salinity}
\end{align}
Thus, $\overline{S_A}$ can change because of $\MS$ change, $M$ change, or both.
For instance, anomalous precipitation freshens the surface ocean because seawater mass increases, without changing the salt mass.

\section{Liquid freshwater volume budget}
\label{sect:LFC}

Liquid freshwater content (LFC) is defined as the volume-integrated fractional salinity anomaly from reference salinity $\Sref$:
\begin{align}
{\rm Liquid~freshwater~content~(LFC)}  = \LFC \equiv \int_V \frac{\Sref - S_A}{\Sref} \; dV .
\label{eqn:lfc}
\end{align}
This definition gives an LFC with units of meters cubed\footnote{Sometimes the integral is over vertical distance, not volume, so liquid freshwater content then has units of meters.}.
To grasp the physical meaning of LFC, consider the following thought-experiment: Begin with a volume $\V $ of incompressible seawater of salinity $\Sref$. 
Replace a volume $\LFC$ of this seawater with freshwater (zero salinity). 
The resulting solution has a mean salinity of $(1 / \V ) \int_V S_A \; dV$.
%Hence, liquid freshwater content is the volume $\LFC$ of freshwater (zero salinity) that hypothetically replaces seawater of salinity $\Sref$ in a volume $V$ of uniform salinity $\Sref$ that gives an average salinity $\int_V S_A \; dV$.
Hence, liquid freshwater content is the volume $\LFC$ of freshwater that replaces seawater in a volume $V$ of salinity $\Sref$ to yield an average salinity $\int_V S_A \; dV$.


LFC (\ref{eqn:lfc}) relates to salt mass (\ref{eqn:salt_mass}) via:
\begin{align}
\LFC  \approx \V  - \frac{\MS}{\rho_0 \Sref}
\label{eqn:lfc_salt_mass_relation}
\end{align}
(recall from section \ref{sect:volume} that $\rho_0$ is the constant characteristic seawater density).
The approximation is an equality under the Boussinesq approximation.

Assuming the Boussinesq approximation\footnote{The Boussinesq approximation is required because, otherwise, volume is not an extensive property. See section \ref{sect:volume} and footnote \ref{fn:extensive}.}, (\ref{eqn:salt_budget}) is
\begin{align}
\frac{d}{d t} \int_V S_A \; dV  & = - \int_\surf  \, \left( {\bf u} S_A + {\bf F}_S \right) \cdot {\hat{\bf n}} \; d \surf , \nonumber \\
& = - \sum_{i=1}^{N} \int_{\surf_i}  \, \left( {\bf u} S_A + {\bf F}_S \right) \cdot {\hat{\bf n}} \; d \surf_i .
\label{eqn:LFC_budget0}
\end{align}
Also, (\ref{eqn:mass_budget_with_faces}) is
\begin{align}
\frac{d \V }{d t} & = \frac{1}{\rho_0} \int_{\surf_s} \Qm \; d\surf_s  - \sum_{i=1}^N \int_{\surf_i} \, {\bf u} \cdot {\hat{\bf n}}  \; d \surf_i ,
\label{eqn:LFC_budget1}
\end{align}
and the LFC tendency from (\ref{eqn:lfc}) is
\begin{align}
\frac{d}{dt} \LFC = \frac{d \V}{d t} - \frac{1}{\Sref} \frac{d}{dt} \int_V S_A \; dV .
\end{align}
Hence, the LFC budget is
\begin{align}
\frac{d}{dt} \LFC &= 
\frac{1}{\rho_0} \int_{\surf_s} \Qm \; d\surf_s  - \sum_{i=1}^N \int_{\surf_i} \, {\bf u} \cdot {\hat{\bf n}}  \; d \surf_i 
+ \frac{1}{\Sref} \left[
 \sum_{i=1}^N \int_{\surf_i}  \, \left( {\bf u} S_A + {\bf F}_S \right) \cdot {\hat{\bf n}} \; d \surf_i 
\right] , \nonumber \\
&= \frac{1}{\rho_0} \int_{\surf_s} \Qm \; d\surf_s  - \sum_{i=1}^N \int_{\surf_i} \, \left[ {\bf u} \left( \frac{\Sref - S_A}{\Sref} \right) - \frac{{\bf F}_S}{\Sref} \right] \cdot {\hat{\bf n}}  \; d \surf_i , 
\label{eqn:LFC_budget} 
\end{align}
This equation resembles the heat budget equations (\ref{eqn:heat_budget_with_faces_Thref0}) and (\ref{eqn:heat_budget_with_faces_Thref}); specifically, $\left(\Sref - S_A \right) / \Sref $ associates with $\Cp \rho_0 \left( \Theta - \Thref \right)$ and ${\bf F}_S / \Sref$ associates with $-\Cp {\bf F}$.
Also, the surface volume flux $Q_m / \rho_0$ associates with the advective surface heat flux $\Qadv$.
In the LFC budget, there is no corollary to the radiative surface heat flux term ${\bf F}_r$.

Similar to the heat budget in section \ref{sect:heat}, from (\ref{eqn:LFC_budget1}) and (\ref{eqn:LFC_budget}) the following conclusions can be drawn:
\begin{enumerate}
\item The LFC budget, meaning $d \LFC / dt$, is ambiguous (depends on the reference salinity) whether seawater mass is constant or not. 
Look at (\ref{eqn:LFC_budget1}): Even if seawater volume is constant, $d \V / dt = 0$, $\LFC$ depends on $\Sref$ through the denominator in the final term. Also, (\ref{eqn:LFC_budget1}) requires the Boussinesq approximation.
Still, for constant seawater volume and under the Boussinesq approximation, $\LFC$ isn't very sensitive to $\Sref$ when $\Sref$ is close to the mean salinity, $\Sref \approx \overline{S_A}$.
\item Computation of advective LFC fluxes, $\LFC_i$, across individual faces, 
\begin{align}
\LFC_i & \equiv \int_{\surf_i} {\bf u} (\Sref  - S_A)/\Sref  \cdot {\hat{\bf n}}  \; d \surf_i  , \nonumber \\
& =  \int_{\surf_i} {\bf u}  \cdot {\hat{\bf n}}  \; d \surf_i  - \frac{1}{ \Sref} \int_{\surf_i} {\bf u} S_A  \cdot {\hat{\bf n}} , \nonumber
\end{align}
depends on the arbitrary reference salinity $\Sref$.
Therefore, $\LFC_i$ has no physical significance.
This statement holds if a non-zero volume flux crosses the face, or not.
Similarly, there is no physical significance to the correlation between the advective LFC flux $\LFC_i$ and the volume flux time series across the face. 
Still, for constant seawater volume and under the Boussinesq approximation, $\LFC_i$ isn't very sensitive to $\Sref$ when $\Sref$ is close to the mean salinity, $\Sref \approx \overline{S_A}$.
Also, the advective LFC flux is a linear combination of the mass flux across the face and the salt flux across the face (under the Boussinesq approximation).
This is a useful relation; for example, to estimate the salt flux from the volume and LFC fluxes across a section.
\item Comparison of the relative importance of the LFC flux across the same face for different times depends on the reference salinity. Therefore, it has no physical significance. 
\item The trend in an advective LFC flux time series across a face has no physical significance. 
\item The extrema (maximum, minimum) in an advective LFC flux time series across a face have no physical significance.
\item Comparison of the relative importance of LFC fluxes across different faces (e.g., Fram Strait versus Davis Strait for the Arctic LFC budget) depends on the reference salinity. Therefore, it has no physical significance.
\item Comparison of advective LFC fluxes $\LFC_i$ across faces from models with each other, and from models with observations, is a legitimate test of model realism, as for heat fluxes.
\item The average salinity over the control volume $V$ is, from (\ref{eqn:average_salinity}),
\begin{align}
\overline{S_A} &= \frac{M_S}{M}  \approx \int_V S_A \; dV  \nonumber
\end{align}
(the final statement is an equality under the Boussinesq approximation).
Therefore, provided the seawater volume $\V$ is constant (and making the Boussinesq approximation), the change in average salinity can be unambiguously determined from the individual terms in (\ref{eqn:LFC_budget}).
This is a useful approach; for example, to interpret observations of LFC fluxes across different straits.
\end{enumerate}

\section{TEOS-10 Freshwater mass budget}
\cite{ioc10} defines a (dimensionless) freshwater mass density as:
\begin{align}
\text{TEOS-10~freshwater~mass~density} = 1 - S_A , \nonumber
\end{align}
where $S_A$ is the absolute salinity (measured in kg/kg).
% See TEOS-10 handbook page 46.
Thus, the TEOS-10 freshwater mass is
\begin{align}
\text{TEOS-10~freshwater~mass} = \int_V \rho \left( 1 - S_A \right) \; dV . \nonumber
\end{align}
In words, the TEOS-10 freshwater mass is the mass in a seawater sample that is 100\% pure freshwater.
In other words, the TEOS-10 freshwater mass plus the salt mass equals the seawater mass.
Therefore, like seawater mass and salt mass (sections \ref{sect:mass} and \ref{sect:salt}), interpretation of the TEOS-10 freshwater mass is unambiguous.
Under the Boussinesq approximation the TEOS-10 freshwater mass equals (minus) the liquid freshwater content (\ref{eqn:lfc}) if $\Sref = 1000$~g/kg.



\section{Proposed best practices}

For reporting flux and budget data, either from field observations or numerical models, the following best practices are proposed:\footnote{Comments welcome!}

For fluxes across individual faces (straits), state:
\begin{enumerate}
\item The face-averaged conservative temperature, absolute salinity, and seawater density timeseries.
\item The mass flux timeseries (the terms in (\ref{eqn:mass_budget_with_faces})) and salt flux timeseries (the terms in (\ref{eqn:salt_budget})).
\item Any tacit assumptions, such as a closed mass budget or the Boussinesq approximation.
\end{enumerate}
Also, optionally, state:
\begin{enumerate}
\item The seawater volume flux timeseries.
\item The advective relative heat flux timeseries $\Heat_i$ and advective LFC flux timeseries $\LFC_i$, as long as at least two reference temperatures and salinities are used. 
\end{enumerate}
For control volumes, state:
\begin{enumerate}
\item The volume-averaged seawater mass, conservative temperature, absolute salinity, and seawater density timeseries. 
\item Any tacit assumptions, such as a closed mass budget or the Boussinesq approximation.
\end{enumerate}
Also, optionally, state:
\begin{enumerate}
\item The seawater volume timeseries.
\item The relative heat content $\Heat$ and LFC timeseries $\LFC$, as long as at least two reference temperatures and salinities are used. 
\end{enumerate}

In all cases, provide access to the original data and the data-processing pipeline.
Fluxes and budgets of trace chemicals are treated analogously to salt mass and salinity.



\bibliographystyle{plainnat}
\bibliography{budgets_theory_refs}

\end{document}  




\section{To do}
To address:
\begin{enumerate}
\item Michael Karcher point about tacit assumption of exiting flux having the reference temperature?
\item Look at \cite{bacon_etal15} and use of face-averaged salinity as the appropriate reference salinity?  \citet{schauer&losch19} dispute the choice of face-averaged salinity as an appropriate reference salinity.  E.g., they point out that two adjacent control volumes with a common boundary have (in general) two different face-averaged salinities.
\end{enumerate}


\section{Other thoughts}
\begin{enumerate}
\item See also \texttt{Flux protocol.odt} from the 2019 Copenhagen ASOF meeting.
\item Check consistency with \cite{piecuch_etal17,schauer&losch19,forget&ferreira19,tesdal&haine20,nguyen_etal21,sanders_etal22}.
\item Heat budgets are somewhat different than salt budgets because (i) temperature is sign indefinite, whereas salinity is positive definite (in other words heat has an arbitrary zero, whereas salt mass does not), (ii) phase changes can occur.
\item In what sense is it legitimate to say something like ``The Arctic Ocean got fresher because of XX\% more runoff, YY\% fresher Bering Strait inflow, and ZZ\% less freezing''?  Are these unambiguous statistics, or do they depend on $\Sref$?  Notice that, by construction, a salt budget will set XX=0, because runoff carries zero salinity.
\item Importance of getting signs right and talking about ``increases'' and ``decreases'' in an unambiguous way. 
\item 28Feb23: Following WebEx with Michael Karcher and Lola Hernandez: What's the best way to trace the individual components of freshwater. E.g., how do we trace river runoff flowing through Fram Strait and Davis Strait?  Michael believes there's no solution to this issue at present. True?  
%Michael also mentions that Ursula and Agniseszka worked out the issues for heat budgets in an earlier paper to \citet{schauer&losch19}. He also says Ursula had demonstration Excel sheets to teach with.
\end{enumerate}




\section{Qiang Wang thought experiment (email 30Oct22)}
\begin{quote}
Considering a lab experiment with a big tank. Four persons are controlling four openings, who can set the outflow or inflow rate as they want, and they can control the inflow temperature (set the temperature in the inflow) and the outflow temperature (for example, changing mixing the water in the big tank in some way). They do not communicate with each other. We want to know the “change” of the “mean temperature” in the tank, and the contribution of each person to the change in the tank. We certainly know how to do the calculation. The water heat flux in each opening has physical meaning.
\end{quote}

It's mysterious how the people can control the outflow temperature without applying heat fluxes. 
Changing the mixing may, or may not, give the required control. 
For example, mixing can only diminish temperature differences, not enhance them.

The relevant equation is (from (\ref{eqn:heat_budget_with_faces_Thref})):
\begin{align}
\frac{d\Heat}{dt} 
& = \int_{\surf_s}  F  \; d\surf_s  - \Cp \sum_{i=1}^4  \rho_i  {\bf u}_i \left( \Theta_i - \Thref \right) \, \Delta S_i  , \nonumber \\
& = \int_{\surf_s}  F  \; d\surf_s  - \Cp \sum_{i=1}^4  \rho_i  {\bf u}_i \Theta_i \, \Delta S_i - \Cp \Thref \frac{dM}{dt} , 
\label{eqn:heat_budget_Qiang_thought_expt}
\end{align}
where $F$ is the radiative and conductive heat flux over the surface of the tank $\Omega _s$, $\Delta S_i$ is the cross-sectional area of each opening, and the outflow volume flux ${\bf u}_i$, temperature $\Theta_i$, and density $\rho_i$ are uniform across the opening.
The mean temperature is
\begin{align}
\overline{\Theta} = \frac{\Heat}{\Cp M} = \frac{ \int_V \rho \left( \Theta - \Thref \right) \; dV}{\int_V \rho  \; dV} = \frac{{\mathcal H}^0}{\Cp M} - \Thref
\end{align}
Given all the measurements, $F, \Theta_i, \rho_i, {\bf u}_i ~~\forall i$ and the value of $\Thref$, the change in heat content can be computed, and hence the change in mean temperature.
However, because $\Thref$ is arbitrary, ``the contribution of each person to the change in the tank'' is ill-defined.
If one doesn't have all the measurements, then the change in heat content can't be computed.

Also, any increase in temperature will tend to warm the tank contents. 
No extra information about the tank or the other openings is required to legitimately say the temperature increase will tend to warm the tank.
But an increase in speed of one inflow has an indeterminate effect without more information about the tank contents and the other openings.


\begin{quote}
Experiments were done many times. Just once we find the water volume in the tank did not change (we do not consider expansion of water due to temperature change here). The reason is by chance the volume fluxes in the 4 gateways sum up to zero. One starts to think: in this case, in the calculation of heat flux through the opening, we can use T-Tref instead of T; summing the heat fluxes over the 4 openings, the net flux can explain the heat change in the tank as well, because the differences associated with Tref cancel out. One thinks further: although I can do this, but the heat flux is fully dependent on Tref. One finds that calculating heat flux in an individual opening by using T-Tref does not make sense. Then one draws the conclusion that heat flux in an individual opening does not exist if the volume transport in an opening is not zero. And one also thinks calculating heat flux using (T-0) does not have sense too, as Tref=0 is still a calculation with reference (just the reference by chance is 0). Actually, one is in the logic trap.
\end{quote}
This doesn't make sense to me.

\begin{quote}
The logic trap started when one found the total volume transport in the four gateways is zero and introduced the reference temperature in the flux calculation. One forgets that this single experiment is not different from all other experiments in which the total volume transport is not zero. Actually, in any experiments, the physical heat flux through each opening exists and can be quantified to explain the change of heat content in the tank. (And it can be compared with surface heat flux if one more person is heating the water above).

In the above explanation/experiments, one can replace temperature with CO2 concentration, or amount of another chemical tracer. It is universal. Maybe one can understand the issue more easily by using a chemical tracer in the tank experiment.

\end{quote}
Is this true?  Is it identical for chemical tracer?

\begin{quote}
To get out of the logic trap, forget the property that the volume transports over 4 openings sum to zero. This property does not matter for the temperature/CO2 flux in each individual opening; and the flux in an individual opening has its physical meaning in nature. (But there is only one physical definition for their flux: no reference is introduced)
\end{quote}

\section{Other thoughts}

Start from salinity and temperature equation, instead of from salt and heat conservation?
\citet{ioc10} say (their equation (A.21.8)):
\begin{align}
\frac{\partial }{\partial t} \left( \rho S_A \right) + \nabla \cdot \left( \rho {\bf u} S_A \right)  = \rho \frac{D S_A}{D t} = - \nabla \cdot {\bf F}^{\rm S} + \rho {\mathcal S}^{S_A}
\end{align}